\documentclass[11pt,table]{beamer}
%Belangrijke packages voor Schrijven van acenten, trema
\usepackage[utf8]{inputenc}
\usepackage[T1]{fontenc}
\usepackage{helvet}
%
\usetheme{Luebeck}%Overall Thema
\usefonttheme{professionalfonts}%Font Thema
\definecolor{sapphire}{rgb}{0.03, 0.15, 0.4}%Kleur Definiëring voor Thema
\usecolortheme{seagull}%Uitvoering thema
%%Navigation bar
\setbeamercolor{section in head/foot}{fg=white,bg=black}

\makeatletter
\setbeamertemplate{headline}{%
    \begin{beamercolorbox}[ht=2.25ex,dp=3.75ex]{section in head/foot}
      \vskip2pt\insertnavigation{\paperwidth}
    \end{beamercolorbox}%
}%
\makeatother
%% Eind Navigation bar

%Een package die mogeljk maakt bestanden uit subdirectories toe te voegen in het document. \subimport{Grafieken/} is een relatieve pad en \import{/c:users/Edon/documenten/...} is absoluut.
\usepackage{import}
%%

% Verwijdering navigatie iconen
%\setbeamertemplate{navigation symbols}{}
%%

%%Toevoeging Frame counter in voet + Herschrijving Code
\expandafter\def\expandafter\insertshorttitle\expandafter{%
   \insertshorttitle\hfill%
   \insertframenumber\,/\,\inserttotalframenumber}
   
%\setbeamercolor{mycolor}{fg=black,bg=title in head}
%
%\defbeamertemplate*{footline}{theme}{%
%\leavevmode%
%\hbox{\begin{beamercolorbox}[wd=.5\paperwidth,ht=2.5ex,dp=1.125ex,leftskip=.3cm plus1fil,rightskip=.3cm]{author in head/foot}%
%    \usebeamerfont{author in head/foot}\hfill\insertshortauthor
%\end{beamercolorbox}%
%\begin{beamercolorbox}[wd=.4\paperwidth,ht=2.5ex,dp=1.125ex,leftskip=.3cm,rightskip=.3cm plus1fil]{title in head/foot}%
%    \usebeamerfont{title in head/foot}\insertshorttitle\hfill%
%\end{beamercolorbox}%
%\begin{beamercolorbox}[wd=.1\paperwidth,ht=2.5ex,dp=1.125ex,leftskip=.3cm,rightskip=.3cm plus1fil]{mycolor}%
%\hfill\insertframenumber\,/\,\inserttotalframenumber
%\end{beamercolorbox}}%
%\vskip0pt%
%}

%
\usepackage[dutch]{babel} %Nederlandse Taal
%%

%%%%% Wiskunde Paketjes
\usepackage{amsmath}
\usepackage{amsfonts}
\usepackage{amssymb}

%%%%Herdefiniëring van de wortel
\usepackage{letltxmacro}
\makeatletter
\let\oldr@@t\r@@t
\def\r@@t#1#2{%
\setbox0=\hbox{$\oldr@@t#1{#2\,}$}\dimen0=\ht0
\advance\dimen0-0.2\ht0
\setbox2=\hbox{\vrule height\ht0 depth -\dimen0}%
{\box0\lower0.4pt\box2}}
\LetLtxMacro{\oldsqrt}{\sqrt}
\renewcommand*{\sqrt}[2][\ ]{\oldsqrt[#1]{#2} }
\makeatother
%%%%

%%%%% Graphs
\usepackage{xcolor}%%Kleurgeving van columnen, rijen en cellen in tabellen. 
\usepackage{graphicx}%%Externe Foto toevoeger
\usepackage{tikz}%% Intern Graph Maker
\usetikzlibrary{mindmap}
\graphicspath{{./Graphics/}}%%Geeft submap van Graphics aan, voor geordend werken.
%%

%%%%%Tabel
\usepackage{array,multirow, booktabs}%%Tabellenmaker 1.Array voor horizontale lijnen en arraystretch, 2. Multirow en Multicolumn voor merging rijen en columnen respectievelijk, 3. Booktabs voor code commands in tabel.
\renewcommand{\arraystretch}{1.5}%Ruimte tussen twee rijen
\setlength{\tabcolsep}{10pt}%Ruimte tussen twee colommen 

%%%Titelpagina
\author{Edon Namani}
\title{Dummy Titel}
\subtitle{Dummy Subtitel}
%\setbeamercovered{transparent} 
%\setbeamertemplate{navigation symbols}{} 
\logo{\includegraphics[width=.05\textwidth]{LogoFysica.pdf}} 
\institute{\includegraphics[width=.2\textwidth]{LogoFysica.pdf}\\
Jouw instituut of bedrijf} 
\date{\today} 
\subject{Professionele Beamer Presentatie} 
%%%
